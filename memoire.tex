\documentclass[12pt,a4paper]{report}

% ============= PACKAGES =============
\usepackage[utf8]{inputenc}
\usepackage[french]{babel}
\usepackage[T1]{fontenc}
\usepackage{graphicx}
\usepackage{geometry}
\usepackage{fancyhdr}
\usepackage{hyperref}
\usepackage{titlesec}
\usepackage{glossaries}
\usepackage{csquotes}
\usepackage{xcolor}
\usepackage{verbatim}
\usepackage{listings}
\usepackage{parskip}
\usepackage[backend=biber,style=numeric,sorting=none]{biblatex}

% ============= CONFIGURATION =============
\geometry{margin=2.5cm}
\pagestyle{fancy}
\fancyhf{}
\fancyfoot[R]{\thepage}
\renewcommand{\headrulewidth}{0pt}

\setlength{\parskip}{0.5em}  % Espace entre paragraphes

% ============= SUPPRESSION COMPLÈTE DE "CHAPITRE X" =============
\titleformat{\chapter}[block]
  {\normalfont\huge\bfseries}
  {}
  {0pt}
  {}
\titlespacing*{\chapter}{0pt}{-20pt}{20pt}

% Glossaire
\makeglossaries

% Bibliographie
\addbibresource{references.bib}

% ============= DÉBUT DU DOCUMENT =============
\begin{document}

\pagenumbering{arabic}

% ============= PAGE DE GARDE =============
\begin{titlepage}
    \centering
    \vspace*{2cm}
    
    {\Huge\bfseries Les valeurs éthiques des ingénieurs dans les organisations\par}
    \vspace{2cm}
    
    {\Large Loïc Weber\par}
    {\large Étudiant ingénieur IDIA\par}
    
    \vfill
    
    {\large Tuteurs entreprise et SSAT :\par}
    \vspace{0.5cm}
    {\large Loïc LABAGNARA\par}
    {\large Emmanuel LIGNER\par}
    
    \vfill
    
    % Logos en bas
    \begin{figure}[h]
        \centering
        \includegraphics[width=0.3\textwidth]{./assets/polytech-nantes.jpg}
        \hspace{2cm}
        \includegraphics[width=0.3\textwidth]{./assets/itii.png}
				\hspace{2cm}
        \includegraphics[width=0.5\textwidth]{./assets/iadvize.png}
    \end{figure}
    
    \vspace{1cm}
    {\large Année 2024-2026\par}
\end{titlepage}

% ============= REMERCIEMENTS =============
\chapter*{Remerciements}
\addcontentsline{toc}{chapter}{Remerciements}

Je tiens à remercier...

Ma référente d'apprentissage Madame Marie Noelle Martin
Mon tuteur pédagogique Monsieur Philippe Leray

\newpage

% ============= SOMMAIRE =============
\tableofcontents
\newpage

% ============= GLOSSAIRE =============
% Définition des termes du glossaire
\newglossaryentry{latex}{
    name=LaTeX,
    description={Système de composition de documents}
}

\newglossaryentry{SAAS}{
    name=SAAS,
    description={Software as a Service}
}

\printglossary[title=Glossaire]
\addcontentsline{toc}{chapter}{Glossaire}
\newpage

% ============= INTRODUCTION =============
\chapter{Introduction}

\section*{Contexte historique}

Historiquement, la notion de valeur était biologique, associée à la survie humaine. Avec l'apparition de la monnaie, elle est devenue un capital (terres, or…). 
Puis, dans l'économie de marché, elle s'est largement financiarisée, perdant une grande partie de son lien avec l'humain.
Aujourd'hui, bien qu'encore financières, les valeurs retrouvent une dimension humaine, devenant également culturelles, morales, éthiques, sociales et environnementales. Cette ré-humanisation des valeurs s'inscrit historiquement après de nombreux événements emblématiques où les valeurs éthiques ont été négligées au profit de la réussite financière, militaire ou technique.

\subsection*{Le cas Monsanto}

Parmi ces événements, on retrouve le cas de Monsanto, une entreprise dont la réputation a été gravement ternie par des accusations d'écocide de par la vente de l'Agent Orange à l'armée américaine pendant la guerre du Vietnam.
Les ingénieurs de Monsanto, ainsi que d'autres entreprises chimiques, savaient que l'Agent Orange était toxique mais ont continué à le produire et à le fournir à l'armée américaine.
Les conséquences de ces actions ont été dévastatrices pour l'environnement et la santé des populations locale, causant des maladies graves et des malformations qui ont fait 3 millions de victimes direct et indirect et a conduit à de nombreuses plaintes contre Monsanto et d'autres entreprises impliquées.

\subsection*{Le Projet Manhattan}

On peut aussi citer le Projet Manhattan lors de la Seconde Guerre mondiale, dirigé par Robert Oppenheimer, qui a conduit au développement de la bombe atomique. Après les bombardements d'Hiroshima et de Nagasaki, de nombreux scientifiques, y compris Oppenheimer, ont exprimé des remords et des dilemmes éthiques concernant leur participation à ce projet.

Ces événements soulèvent d'importantes questions sur la responsabilité éthique des ingénieurs et des organisations.

\section*{Motivation personnelle}

En tant que futur ingénieur, je ressens souvent une forme de tension entre mes valeurs et la réalité du monde du travail.
Certaines perspectives de carrière, comme celles proposées dans la finance ou l’assurance, me semblent complètement contradictoires avec mes valeurs.
Mais pour avoir beaucoup discuté avec d'autres ingénieurs, je sais que ce sentiment n'est pas partagé par tous.

L’exemple des ingénieurs d’AgroParisTech, qui ont publiquement exprimé leur désaccord avec les valeurs de certaines industries, me touche personnellement.
Ces jeunes ingénieurs, tout juste sortis de l'école, refusent de se soumettre à une logique purement économique ou industrielle au détriment de l’éthique.

\section*{Contexte professionnel}

Actuellement alternant ingénieur Platform, mon métier consiste à construire une plateforme de développement pour les ingénieurs développeurs.
Dans cette mission, je suis donc responsable du bon fonctionnement et de l'amélioration continue des outils de travail de ces ingénieurs, afin d'augmenter la vélocité à laquelle ils créent et déploient de nouvelles fonctionnalités à nos clients.
J'occupe ce rôle au sein de l'entreprise iAdvize.

iAdvize est un éditeur de logiciel (\gls{SAAS}) de taille moyenne (~150 employés).
Elle occupe une place importante dans l'écosystème du E-commerce européen. Avec des clients dans une dizaine de pays, elle a pour objectif de révolutionner l'expérience d'achat en ligne en y intégrant une dimension conversationnelle.

iAdvize, comme toute entreprise moderne, possède des valeurs.

Mais quelles sont ces valeurs et comment s'articulent-elles autour des valeurs éthiques individuelles des ingénieurs qui conçoivent les systèmes que iAdvize vend ?

\section*{Problématique}

Ainsi nous tenterons de répondre a la problématique suivante :

\begin{center}
	\textbf{Quelles sont les valeurs éthiques des ingénieurs et quelle est leur place dans les organisations dans lesquelles ils travaillent ?}
\end{center}

Cette problématique est donc le résultat de plusieurs itérations de travail, commençant par me questionner sur le bonheur au travail et sur le masque sociale au travail puis sur les enjeux et valeurs éthique de mon entreprise et enfin sur la place des ingénieurs et de leurs valeurs.

\section*{Hypotèses}

A partir de cette problématique, nous axerons notre mémoire sur les hypothèses suivantes :

\begin{itemize}
	\item Les ingénieurs ont besoin d'associer les objectifs de leur entreprise à des valeurs éthiques pour éviter un sentiment d'aliénation au travail. \\
	\item Soucieux des enjeux écologiques et sociaux actuels, les jeunes ingénieurs tendent à se détourner de certains secteurs pour privilégier des organisations qui tentent de répondre concrètement aux problématiques actuelles et qui mettent l'humain au coeur de leur démarche.\\
	\item Les ingénieurs ont le besoin d'être tenu dans un cadre éthique pour éviter que leurs créations techniques ne soient détournées à des fins contraires à l'intérêt général.\\
\end{itemize}

% ============= DÉVELOPPEMENT =============
\chapter{Développement}

\textcolor{red}{TODO: Poser le fond du contexte d'étude et amenée la problématique}

\section{Problématique}

Quelles sont les valeurs éthiques des ingénieurs et quelle est leur place dans les organisations dans lesquelles ils travaillent ?

\textcolor{red}{TODO: Définition des concepts de la problématique}

\section{Hypothèses}

\begin{enumerate}
    \item Les ingénieurs ont besoin d'associer les objectifs de leur entreprise à des valeurs éthiques pour éviter un sentiment d'aliénation au travail.
    
    \item Soucieux des enjeux écologiques et sociaux actuels, les jeunes ingénieurs tendent à se détourner de certains secteurs pour privilégier des organisations qui tentent de répondre concrètement aux problématiques actuelles et qui mettent l'humain au cœur de leur démarche.
    
    \item Les ingénieurs ont le besoin d'être tenus dans un cadre éthique pour éviter que leurs créations techniques ne soient détournées à des fins contraires à l'intérêt général.
\end{enumerate}

\section{Le Plan d'action (Protocole de recherche)}

Dans l'objectif de récolter des données qualitatives et quantitatives afin d'infirmer, de réfuter ou de nuancer nos hypothèses, nous utiliserons deux méthodes :

\subsection{La grille de questionnaire pour les entretiens informels}

Sur la base de la grille de questionnaire suivante :

\subsubsection*{Hypothèse 1}

\begin{quote}
\textit{Les ingénieurs ont le besoin d'être tenus dans un cadre éthique pour éviter que leurs créations techniques ne soient détournées à des fins contraires à l'intérêt général.}
\end{quote}

\paragraph{Question 1.1 :}
Est-ce qu'il y a des parties de ton travail que tu trouverais impossibles à expliquer à des gens non techniques ? Si oui, pourquoi ? Et est-ce que ça te pose un problème ?

\paragraph{Rebond 1 :}
\begin{itemize}
    \item Si oui $\Rightarrow$ Si personne ne comprend à part toi ou un groupe restreint de personnes, n'est-ce pas dangereux éthiquement parlant ? Comment tu en parles démocratiquement avec d'autres personnes ?
    \item Si non $\Rightarrow$ Comment tu fais pour formaliser les concepts / implémentations très techniques ?
\end{itemize}

\paragraph{Objectifs 1 :}
\begin{itemize}
    \item Évaluer si la personne perçoit une différence entre le langage technique et le langage commun.
    \item Identifier si elle a conscience des limites de la traduction.
    \item Observer si elle relie cette difficulté à des risques éthiques.
\end{itemize}

\paragraph{Question 1.2 :}
Est-ce que tu as déjà travaillé sur un système sur lequel tu ne maîtrisais pas tous les impacts éthiques (sociaux, politiques, environnementaux) ?

\paragraph{Rebond 1.2 :}
\begin{itemize}
    \item Si oui $\Rightarrow$ Est-ce que tu as cherché à prendre du recul sur ces impacts ?
    \item Si non $\Rightarrow$ Comment tu fais pour t'assurer que les systèmes que tu conçois sont éthiques ?
\end{itemize}

\paragraph{Objectifs 1.2 :}
\begin{itemize}
    \item Savoir si la personne a déjà été confrontée à une dissonance entre son travail technique et ses implications.
    \item Comprendre ses mécanismes de responsabilité (ex.~: revues éthiques, discussions en équipe, auto-censure).
    \item Comprendre si la personne pense aux impacts des systèmes qu'elle crée et surtout si elle se questionne sur les enjeux éthiques des solutions qu'elle conçoit.
\end{itemize}

\paragraph{Question 1.3 :}
Est-ce que tu te sens protégé par le cadre éthique de ton entreprise lors de la création de systèmes techniques ? Par exemple, si tu avais un doute sur les conséquences d'un système que tu développes, est-ce que tu saurais à qui t'adresser ou comment agir ?

\paragraph{Objectifs 1.3 :}
\begin{itemize}
    \item Savoir si l'ingénieur se sent soutenu par son entreprise quand des dilemmes éthiques surgissent.
    \item Identifier si des mécanismes concrets existent ou si c'est flou/inexistant.
    \item Comprendre si la personne a déjà été confrontée à des situations où l'éthique était en tension avec les objectifs techniques ou commerciaux.
\end{itemize}

\subsubsection*{Hypothèse 2}

\begin{quote}
\textit{Les ingénieurs ont besoin d'associer les objectifs de leur entreprise à des valeurs éthiques pour éviter un sentiment d'aliénation au travail.}
\end{quote}

\paragraph{Question 2.1 :}
Quand tu regardes les projets sur lesquels tu travailles ou les objectifs de ton entreprise, est-ce que tu as l'impression qu'ils s'alignent avec tes valeurs éthiques personnelles ?

\paragraph{Rebond 2.1 :}
\begin{itemize}
    \item Si oui $\Rightarrow$ Est-ce que tu as choisi ces missions / entreprise pour cette raison ? $\Rightarrow$ Comment tu fais si demain on te propose un travail qui n'est pas aligné avec tes valeurs ?
    \item Si non $\Rightarrow$ Est-ce que cela te pose problème pour travailler ? Si oui, quels problèmes ? Ou bien arrives-tu à faire abstraction ? Et si oui, comment ?
\end{itemize}

\paragraph{Objectifs 2.1 :}
\begin{itemize}
    \item Identifier le niveau d'alignement valeurs/entreprise et surtout s'il existe une injonction des valeurs et comment ils réagissent.
\end{itemize}

\paragraph{Question 2.2 :}
\begin{itemize}
    \item Est-ce que tu as une vision claire de l'utilité sociale ou collective de ce que tu produis ?
    \item Si tu devais expliquer à un proche à quoi et à qui sert ton travail, comment tu le formulerais ?
\end{itemize}

\paragraph{Rebonds 2.2 :}
\begin{itemize}
    \item Si c'est flou $\Rightarrow$ Est-ce que ce manque de visibilité sur l'utilité de ton travail te dérange ? Ou bien tu te concentres uniquement sur la partie technique ?
    \item Si la réponse est précise $\Rightarrow$ Est-ce que cette clarté sur l'utilité t'aide à te motiver au quotidien ? Ou est-ce que tu as déjà douté, même sur des projets qui devraient avoir du sens ?
\end{itemize}

\paragraph{Objectifs 2.2 :}
\begin{itemize}
    \item Savoir si l'ingénieur perçoit son travail comme un moyen ou une fin.
    \item Identifier si le manque de visibilité sur l'impact génère de la frustration.
    \item Révéler des mécanismes de défense contre le sentiment d'aliénation.
\end{itemize}

\subsection{Recherche d'études, témoignages, articles en ligne}

\begin{itemize}
    \item Bibliothèque numérique du Cairn
    \item Témoignages vidéo d'ingénieurs militants
    \item Sites gouvernementaux
    \item Articles Le Monde, France Inter...
\end{itemize}

\section{Situations étudiées}

\textcolor{red}{TODO: les évolutions du sujet}

\section{Le concept d'éthique}

\subsection{L'éthique au sein d'une organisation}

L'éthique est un ensemble de principes moraux qui constitue une base pour la conduite humaine, incluant des aspects sociaux, environnementaux, humains, animaux et spirituels. Dans le contexte de l'entreprise, l'éthique correspond à l'application de ces principes à la conduite des affaires, englobant toutes les décisions et comportements discrétionnaires non réglementés. Les valeurs éthiques d'une organisation sont ainsi le résultat d'un équilibre entre les convictions individuelles des dirigeants, les attentes des employés et les pressions externes exercées sur l'entreprise.

\begin{quote}
    \textit{Le sujet de l'éthique d'entreprise n'est plus une mode pour le management, mais bien un mode de management.}
\end{quote}
\hfill --- Orse, \textit{Éthique, responsabilité et stratégie d'entreprise.}

Observer les valeurs éthiques d'une entreprise permet de mettre en évidence ses points forts ainsi que sa direction générale. Ces valeurs définissent les croyances, les principes et la culture interne de l'organisation, tout en influençant fortement son image externe.

\begin{quote}
    \textit{Réfléchir à la notion de valeur en entreprise permet de rendre compte des atouts et attraits de l'entreprise au présent, mais aussi et surtout de rassurer les employés, managers et investisseurs sur sa pérennité.}
\end{quote}
\hfill --- Valérie Lejeune, \textit{Tendances économiques et sociales de la valeur en entreprise}

\subsubsection{Comment sont trouvées les valeurs éthiques dans une organisation ?}
Les valeurs éthiques d'une entreprise émergent d'un équilibre complexe entre des facteurs internes et externes. Parmi les pressions externes, on peut citer :
\begin{itemize}
    \item Les lois et politiques publiques, telles que la loi PACTE du 22 mai 2019 ou le plan France 2030, qui visent à favoriser une vision stratégique à long terme pour les entreprises.\\
    \item Les évolutions sociétales et les attentes des consommateurs pour des pratiques durables, qui jouent un rôle central dans la définition et l'adoption de ces valeurs.\\
    \item Les investisseurs, de plus en plus soucieux de critères environnementaux, sociaux et de gouvernance (ESG), influencent également les choix des organisations.\\
    \item Le recrutement de nouveaux employés, notamment parmi les jeunes générations, qui accordent une importance croissante à l'éthique, à la diversité et à l'impact sociétal des organisations.\\
\end{itemize}

\subsubsection{La théorie des parties prenantes : un cadre pour l'éthique organisationnelle}

La théorie des parties prenantes, popularisée par Freeman (1984), propose une vision élargie de l'entreprise, où celle-ci n'est plus considérée comme une entité centrée uniquement sur la maximisation des profits pour les actionnaires, mais comme une institution devant répondre aux attentes de multiples acteurs (salariés, clients, fournisseurs, société civile, etc.). Comme le souligne Mercier (2010), cette approche vise à « identifier et organiser les responsabilités de l’entreprise vis-à-vis des différents groupes qui y contribuent » (p. 143).

\begin{quote}
    \textit{L’entreprise est devenue une constellation d’intérêts plutôt qu’un instrument aux mains d’un seul individu.}
\end{quote}
\hfill --- Berle \& Means, 1932, cités par Mercier, 2010, p. 146

Cette théorie introduit une dimension normative : l’entreprise a une responsabilité éthique envers tous les acteurs affectés par ses activités, au-delà des obligations légales. Clark (1916) soulignait déjà que :
\begin{quote}
    \textit{La responsabilité de l’entreprise doit prendre en compte les conséquences connues de ses activités économiques, qu’elles soient ou non reconnues par la loi.}
\end{quote}
\hfill --- Clark, 1916, cité par Mercier, 2010, p. 145

Cependant, la théorie des parties prenantes fait face à des critiques majeures (Mercier, 2010) :
\begin{itemize}
    \item Ambiguïté du concept : Qui est une "partie prenante" ? Freeman (1984) propose une définition large (tout acteur ayant un "intérêt"), mais cela pose des problèmes pratiques, comme l'inclusion d'acteurs controversés (ex. : terroristes, pollueurs).\\
    \item Hiérarchisation des intérêts : Comment arbitrer entre les attentes contradictoires des différentes parties prenantes ? Dodd (1932) suggérait déjà que les dirigeants doivent équilibrer les droits des actionnaires, des salariés et du public, mais sans critère clair de priorisation.\\
    \item Risque de "managérialisation" : La théorie peut être instrumentalisée pour justifier des décisions discrétionnaires des dirigeants, sans réelle prise en compte des intérêts des parties prenantes (Jensen, 2002).\\
\end{itemize}

\subsubsection{La hiérarchie des valeurs chez iAdvize : au-delà des valeurs affichées}
iAdvize affiche officiellement quatre valeurs :
\begin{enumerate}
    \item \textit{Fun as a must have} (bonheur au travail),\\
    \item \textit{Committed for better} (inclusion, égalité au travail),\\
    \item \textit{Entrepreneur Spirit} (autonomie, flexibilité et initiative),\\
    \item \textit{Learning organization} (apprentissage et développement personnel).\\
\end{enumerate}

Cependant, il existent des valeurs éthiques qui dépassent le cadre officiel. Par exemple, l'engagement écologique est une valeur forte chez iAdvize, que j'ai pu observer de par :
\begin{itemize}
    \item La présence d'une dizaine d'affiches sur l'environnement dans les bureaux,
    \item L'organisation de la "Fresque du climat", un atelier de trois heures suivi par tous les employés,
    \item Des actions volontaires des employés, comme la plantation d'arbres.
\end{itemize}

\textbf{C'est un exemple d'observation personnelle, je suis convaincu qu'il y a de nombreuses autres valeurs éthiques qui dépassent le cadre officiel communiqué par iAdvize}

Cette situation met en lumière une limite de la théorie des parties prenantes : les entreprises sélectionnent et mettent en avant certaines parties prenantes (ex. : employés, clients) au détriment d'autres (ex. : environnement, générations futures). Chez iAdvize, l'écologie, bien que non formalisée comme une valeur officielle, est une partie prenante implicite, ce qui indique que l'entreprise a une responsabilité éthiques plus large que celle quelle communique officiellement.

\subsection{L'éthique au sein de l'individu}

L'éthique de l'ingénieur ne se limite pas à un ensemble de règles morales personnelles. Chaque ingénieur est d'abord un citoyen responsable, qui relie les sciences, les technologies et la communauté humaine. C'est ce que rappelle la Charte d'éthique de l'ingénieur publiée par l'IESF :

\begin{quote}
	\textit{L'ingénieur doit inscrire ses actes dans une démarche de développement durable, faire prendre conscience de l'impact de la technique sur l'environnement et mettre ses compétences au service du bien commun.}
\end{quote}
\hfill --- \textit{Charte éthique de l'ingénieur et scientifique de France IESF}

Cette charte, considérée comme une profession de foi pour les ingénieurs et scientifiques de France, reconnaît que le progrès technique n'est pas neutre (comme vu avec le projet Manhattan). Les avancées qu'il apporte à la société peuvent être porteuses de risques si elles ne sont pas encadrées éthiquement, en effet puisque,
\begin{quote}
Inventer le train, c'est inventer le déraillement, inventer l'avion c'est inventer le crash
\end{quote}
\hfill --- Paul Virlio

Les ingénieurs ont un rôle essentiel et double : maîtriser les technologies au service de la communauté humaine et diffuser une information claire sur leurs limites, leurs risques et leurs avantages.

\subsubsection{L'injonction de valeurs au travail}

En 2016, la Dares poste une étude sur 22~895 actifs révélant que six actifs occupés sur dix signalent être exposés à des conflits de valeurs dans leur travail.

La plus fréquente concerne deux actifs occupés sur dix, qui vivent des conflits éthiques car ils doivent faire des choses qu'ils désapprouvent, même si par ailleurs ils ont les moyens de bien faire leur travail.

Les autres situations touchent chacune un actif occupé sur dix. Certains doivent faire un travail qu'ils jugent en grande partie inutile, ce à quoi s'ajoute, pour d'autres, l'absence de fierté du travail bien fait. D'autres encore estiment manquer de moyens pour bien faire leur travail mais se sentent malgré tout fiers du résultat. Le dernier cas concerne les personnes qui cumulent la plupart des conflits de valeur.

Les salariés les plus exposés aux conflits de valeurs et à leur cumul déclarent plus fréquemment une santé physique et mentale dégradée.

\begin{center}
\textit{(Voir figure~1 en annexes)}
\end{center}

Si maintenant on s'intéresse aux informaticiens dans cette étude, on remarque que les techniciens de l'informatique sont associés à deux profils opposés :
\begin{itemize}
    \item d'un engagement fort entravé par le manque de moyens.
    \item soit d'un désengagement par perte de sens et de qualité malgré des conditions matérielles correctes.
\end{itemize}

De même, les ingénieurs en informatique apparaissent dans la catégorie de ceux qui ressentent leur travail comme peu utile, ce qui révèle un sentiment de déconnexion entre les tâches réalisées et leur finalité perçue.

Malgré des conditions matérielles souvent adéquates (outils informatiques, collègues disponibles, autonomie), ces cadres peuvent souffrir de frustration liée à l'absence de reconnaissance ou au manque d'impact perçu.

L'étude de la Dares montre alors que ces tensions les exposent à des risques accrus de troubles du sommeil, de stress et de baisse du bien-être psychologique, surtout lorsqu'ils se retrouvent dans des situations où la qualité doit être sacrifiée ou lorsque leur utilité sociale semble floue.

\textcolor{red}{TODO: APPUYER AVEC DES ENTRETIENS}

\subsubsection{L'importance d'un cadre éthique pour les ingénieurs}

\begin{quote}
	\textit{Une des raisons profondes de notre incapacité à comprendre est liée, selon Hannah Arendt, à l'usage d'un « langage » de symboles mathématiques. Au début, écrit-elle, cet usage avait pour but de simplifier la communication en concevant des abréviations pour des propositions d'énoncés formulables par la parole. Il « contient à présent des propositions absolument intraduisibles dans le langage » (Arendt, 1961/1958~: 3).
Ainsi, si les scientifiques n'ont pas refusé de développer des armes nucléaires, ce n'est pas d'abord, selon elle, par manque de caractère ou de courage ou encore par naïveté, parce qu'« ils n'auraient pas compris qu'une fois ces armes inventées, ils seraient les derniers consultés sur leur emploi ». 
Ils n'ont pas refusé de poursuivre leur recherche parce qu'« ils évoluaient dans un monde où le langage a perdu son pouvoir. Et toute action de l'homme, tout savoir, toute expérience n'a de sens que dans la mesure où l'on peut en parler ».}
\end{quote}
\hfill --- Christelle Didier, \textit{Figure inspirante pour la formation en éthique des ingénieurs}

Ici, on peut voir de manière flagrante une analogie avec les langages de programmation, qui, comme le langage mathématique, contiennent des propositions intraduisibles dans le langage. Exemple extrait du noyau Linux (projet open source qui constitue la base de l'extrême majorité des systèmes d'exploitation utilisés, notamment les serveurs, les appareils mobiles et les systèmes embarqués) :

\begin{verbatim}
BPF_CALL_3(bpf_probe_read_compat_str, void *, dst, u32, size,
   const void *, unsafe_ptr)
{
  if ((unsigned long)unsafe_ptr < TASK_SIZE) {
    return bpf_probe_read_user_str_common(dst, size,
      (__force void __user *)unsafe_ptr);
  }
  return bpf_probe_read_kernel_str_common(dst, size, unsafe_ptr);
}
\end{verbatim}

Ainsi, l'ingénieur informaticien peut se retrouver concepteur d'un système dont il ne peut comprendre pleinement les conséquences car il évolue dans un monde où le langage a perdu son pouvoir. Cela souligne l'importance d'un cadre éthique pour les ingénieurs en informatique, car ils peuvent créer des systèmes complexes sans toujours comprendre les implications éthiques de leur travail.

Et ce problème est toujours d'actualité avec plus récemment les majeures avancées dans le développement des algorithmes d'apprentissage automatique utilisés notamment dans la reconnaissance faciale ou les systèmes de recommandation. Ces systèmes peuvent avoir des implications sociales et politiques considérables, comme la surveillance de masse ou encore la propagation de fausses informations...

En questionnant les ingénieurs informaticiens de mon entreprise, a la question 1.1 "Est-ce qu'il y a des parties de ton travail que tu trouverais impossibles à expliquer à des gens non techniques ?" 
la réponse était très unanime, Oui la majorité de mon travail technique est inexpliquable au gens non technique, déjà car cela demanderais un investissement colosale en temps a un non-technique pour avoir les bases mais aussi et surtout car les non-technique n'ont pas envie de faire l'effort de comprendre non plus. Et tout ça malgrès tout les efforts que l'on peut faire pour expliquer le plus trivialement possible. 

% ============= CONCLUSION =============
\chapter{Conclusion}

Votre conclusion...

% ============= BIBLIOGRAPHIE =============
\printbibliography[heading=bibintoc,title=Bibliographie]

% ============= ANNEXES =============
\appendix

\chapter{Citations et références}
Ici vous pouvez détailler vos citations...

\chapter{Figures}
\begin{figure}[h]
    \centering
    \includegraphics[width=0.9\textwidth]{./assets/conflits-de-valeurs-au-travail-graphique.png}
    \caption{Figure annexe 1}
\end{figure}

\chapter{Autres annexes}
Contenu supplémentaire...

\end{document}
